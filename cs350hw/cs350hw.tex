\documentclass[12pt,letterpaper]{article}
\usepackage[parfill]{parskip}

\begin{document}

\section*{CTPS-350 | Homework-02 }
\subsection*{Charles Norden, \#011606177 }

\subsection*{PROBLEM 1:}
This section includes the text and paragraphs required for the first homework problem. Today I want to test
out how the vim text editor will play with the latex compiler. In the vim editor, I usually set the line 
length to be 110 characters; at the 111th character, a new line will be automatically inserted.

Now I just tried compiling the doc and the produced pdf file displayed everything perfectly as I wanted.
Apparently, a line between each block of text is considered a paragraph separator. Otherwise, the compiler
just assumes that it's a space and continues as a whole paragraph.

For example, let's try again with the same text but with just one newline char instead of two.

This section includes the text and paragraphs required for the first homework problem. Today I want to test
out how the vim text editor will play with the latex compiler. In the vim editor, I usually set the line 
length to be 110 characters; at the 111th character, a new line will be automatically inserted.
Now I just tried compiling the doc and the produced pdf file displayed everything perfectly as I wanted.
Apparently, a line between each block of text is considered a paragraph separator. Otherwise, the compiler
just assumes that it's a space and continues as a whole paragraph.
\pagebreak

\subsection*{PROBLEM 2:}
In this section, I want to see if the title will be pushed up further if I try to fill the page with more
text. I certainly don't want it to occupy only the center of the page because that would be a waste of page
estate.

Let's also add some mathematical equation in here to expand the size of the content:

\[ \sum_{i=1}^{N} \frac {i^2} {N} \]

This is another equation:

\[ \lim_{x \to \infty } \frac{x^2}{log_2 x} \]

\end{document}
